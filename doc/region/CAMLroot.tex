\documentclass[a4paper]{easychair}
% Available options:
%   EPiC, EPiCempty, debug, verbose, notimes, withtimes, a4paper, letterpaper

\usepackage[utf8]{inputenc}
\usepackage{xcolor}
\usepackage{listings}
%\usepackage{amsmath}
%\usepackage{amsthm}
%\usepackage{amssymb}
%\usepackage{bussproofs}
\usepackage{hyperref}
%\usepackage{enumitem}

% Boxed figures
\usepackage{float}
\floatstyle{boxed}
\restylefloat{figure}
\usepackage{kpfonts}

\definecolor{bg}{rgb}{0.95,0.95,0.95}
%\setminted[ocaml]{bgcolor=bg}
\newcommand{\caml}[1]{\smash{\lstinline{#1}}}

\newcommand{\ind}[0]{\ensuremath{\mathtt{Ind}}}
\newcommand{\sep}[0]{\ensuremath{\mathtt{Sep}}}
\newcommand{\deepsep}[0]{\ensuremath{\mathtt{Deepsep}}}
\renewcommand{\vec}[1]{\ensuremath{\overline{#1}}}

\newcommand{\of}[0]{\ensuremath{\text{ of }}}

%\newcommand{\guard}[2]{#1 \Rightarrow #2}
\newcommand{\guard}[2]{#2 \restriction #1}
\newcommand{\tyeq}[2]{(#1 = #2)}

\newcommand{\bnfeq}{\mathrel{::=}\;}
\newcommand{\bnfor}{\mathrel{\vert}}

\newcommand{\mcompsymbol}{\circ}
\newcommand{\mcomp}[2]{#1 \mathrel{\mcompsymbol} #2}

\newcommand{\subsymbol}{\triangleleft}
\newcommand{\sub}[2]{#1 \mathrel{\subsymbol} #2}

\newcommand{\vdashdecl}{\vdash_{\mathsf{decl}}}

\newcommand{\builtin}[1]{\mathtt{#1}}

\usepackage{natbib}
\bibliographystyle{plainnat}
\citestyle{authoryear}

\definecolor{britishracinggreen}{rgb}{0.0, 0.26, 0.15}
\definecolor{blue(pigment)}{rgb}{0.2, 0.2, 0.6}

\lstdefinestyle{C++}{
  language=C++,
  morekeywords={value,CAMLprim,CAMLreturn,CAMLlocal1,CAMLparam0,CAMLparam2}
}

\lstset{
  basicstyle=\ttfamily,breaklines=true,
  language=[Objective]Caml,
  morekeywords={external},
  commentstyle=\em\color{blue(pigment)},
  stringstyle=\color{britishracinggreen},
}

% A bit hackish, but takes too much space otherwise.
% \setlist[itemize]{noitemsep, topsep=4pt}

\title{CAMLregion: revisiting the OCaml FFI}
\titlerunning{CAMLregion: revisiting the OCaml FFI}

\author{
  Frédéric Bour
}
\authorrunning{Bour}

\institute{
  Facebook --
  \email{def@fb.com}
}

\begin{document}

\maketitle

\section{Introduction}

Writing code bridging C libraries to OCaml code is a difficult task.
While writing \href{https://github.com/let-def/cuite}{Cuite}, an OCaml
library that interfaces the Qt tool-kit, we discovered a few idioms that
help to keep this plumbing simple to reason about.

Qt is a C++ framework that enables writing portable user interfaces.
User interfaces are challenging to write because they involve complex
lifetimes and control flow: data is described as a dynamically changing
graph of components, control can jump back-and-forth between user code
and library code, different tasks can run concurrently, etc.

Interfacing with OCaml means exporting all these features while abiding
by OCaml \& Qt rules about memory management. By revisiting a few
assumptions of the OCaml GC interface, we believe that the CAMLregion 
approach makes this work more tractable and easier to debug.

\subsection{Our approach: focusing on roots}

We observed that the less natural part of OCaml interaction from C code
was the manipulation of roots and the lifetime of OCaml values. The C
programmer is accustomed to manual memory management: by explicitly
creating and destroying pieces of memory or by tying the variable
lifetime to the scope. While syntactically the management of OCaml
memory seems to fall into these cases, it is actually much more subtle.

If the OCaml garbage collector triggers at the wrong time, (1) a value
can get moved, (2) a piece of OCaml memory that is locally referenced
but not registered as a root can get collected. If (1) happens in the
middle of a \emph{sequence point} (TODO: reference the definition of C
sequence point?) where the same value has been read, this results in an
undefined behavior of the C language. In practice, the lifetime of the
value gets disconnected from the lifetime of the variable that holds it.
This behavior completely contradicts the intuition of the C developer
who is not used to distinguishing a variable from its contents.

Fortunately (1) can be addressed by a slight, almost mechanical, change
to the C API of the garbage collector. By preventing allocating
functions from having \texttt{value} return type (ideally by preferring
them to be \texttt{void} functions), this class of error can be
ruled-out.

And while the programmer could be blamed for not following the rules in
(2), we propose an alternative root management strategy that is more in
line with usual C practices and should prevent that kind of mistake.

\subsection{Contributions}

We claim the following three contributions:

\begin{itemize}
\item a general design principle for OCaml FFI functions, working with value
      pointers rather than plain values, that resolves a class of FFI bugs and
      integrates well with existing FFI code,
\item an alternative approach for the management of local roots that trades
      some performance for ease of use and safety,
\item a reusable, open-source, library that implements both of those.
\end{itemize}

While these were developed in the context of Cuite, the Qt binding, this
paper solely focuses on the management of OCaml memory from C code.
However to illustrate its applicability to realistic code, we describe
how our proposed API behaves in complex situations -- involving
callbacks, exceptions and multi-threaded code.

\section{The original OCaml-C FFI}

The OCaml FFI lets the developer manipulate OCaml values from another
programming language. A C library provided by the OCaml distribution
exposes the primitive operations to achieve that.

This library helps accomplishing two main tasks:

\begin{enumerate}
\item constructing and deconstructing OCaml values, interpreting them in a
      meaningful way from the C language (for instance by mapping
      back-and-forth between OCaml and C representations of integers, of
      strings, etc);
\item cooperating with the OCaml garbage collector, or GC, the runtime
      service that takes care of managing OCaml memory.
\end{enumerate}

Even though both tasks are much more difficult than when working from
within OCaml code (the typechecker will not help you in foreign code),
it is at least possible to reason locally, in a compositional way, about
OCaml values -- task (1). For instance, building nested tuples just
involves repeatedly building flat tuples.

The same cannot be said about task (2). The GC needs to know about all
OCaml values that are manipulated from C code, and can look at them at
almost any moment. These restrictions are not natural while programming
in C and can lead to subtle bugs that are hard to discover.

Most of the time the GC will not do any work, preferring to wait for a
batch of work that is big enough to amortize its overhead. As such an
improper use of the OCaml API can go unnoticed for a long time. But even
once harm has been done, it might just lead to a corruption of the OCaml
heap that will affect an unrelated piece of code and fail much later in the
program.

GC bugs combine two nasty properties: they cannot be studied in
isolation and they trigger depending on a complex set of conditions that
cannot be inferred by solely looking at the buggy code.

On the other hand, the OCaml FFI API enjoys a remarkably low overhead:
the restrictions are difficult but lead to a cheap and portable
interface with the OCaml runtime. This made OCaml applicable to domains
where connecting to a foreign programming language is generally
considered too expensive {[}TODO reference Sundial{]}.

We propose to explore a different trade-off in the design space of FFI
API: providing a safer and more convenient API by giving up some of the
performance.

This might actually be a generally more useful position: many mainstream
languages adopted heavier FFIs by default (Lua, Python, Java JNI, Go),
optionally allowing to resort to a lower-level one for performance critical
code (ctypes from LuaJit).

But before trying to build an alternative to the FFI abstraction, we will take
a closer look at the restrictions imposed by the GC.

\subsection{Value representation}

From the C programming language, all OCaml values are represented by the
\texttt{value} type. It is a signed integer of the same size as a
pointer of the host system (in practice, 32 or 64 bits). Values of this
size are called ``words''.

The least significant bit is reserved to help the GC traverse the OCaml
heap:

\begin{itemize}
\item if it is set, the value is said to be immediate and the remaining bits
      are directly interpreted: as a 31-bit or 63-bit integer for the
      \texttt{int} OCaml type, or as a unique pattern of bits for constant
      variants and polymorphic variant constructors
\item if it is not set, the value is interpreted as a pointer to a
      ``block'', a piece of memory provided by the runtime and guaranteed to
      be aligned.
\end{itemize}

Blocks are preceded by a header that determines their ``tag'' and their
size. The tag will determine how to interpret the contents of the block.
For common OCaml values, such as algebraic data, records, tuples or
arrays, blocks are made of other values.

\subsection{Traversal and compaction}

Under certain conditions, the OCaml GC might need to traverse the heap.
The basic operation is to find which blocks are reachable from a value.

Depending on the tag, the OCaml GC will decide whether a block is made
of other \texttt{value}s (that, in turn, can be immediate or pointer to
blocks) or just of an opaque chunk of memory that does not need to be
scanned.

By repeating this operation, the GC can traverse the whole heap. If
necessary it might also decide to move some blocks.

Moving blocks is more demanding than mere traversal: the GC not only
needs to know all values referenced from C code, it also needs to be
able to update them. The C compiler needs to be aware that all OCaml
values have to be reloaded when such an operation can happen, as
previous values might have been invalidated.

\subsection{The memory management macros.}

A few C macros are provided by the OCaml runtime to implement foreign
features. As a guiding example, here is a simple function that takes two
values and makes a pair out of them:

\begin{lstlisting}[]
(* The OCaml version *)
let mk_pair_ocaml x y = (x,y)

(* An external function, that we will implement in C *)
external mk_pair_c : 'a -> 'b -> 'a * 'b = "mk_pair_c_impl"
\end{lstlisting}

The string after the external declaration is the name of the C function
that implements the functionality:

\begin{itemize}
\item when \texttt{mk\_pair\_c} is applied, the OCaml compiler generates a
      call to \texttt{mk\_pair\_c\_impl} symbol,
\item a C file must define a function of this name
\item the C compiler turns it into an object file defining this symbol,
\item the system linker connects the two together.
\end{itemize}

The corresponding C code looks like this:

\begin{lstlisting}[]
CAMLprim
value mk_pair_c_impl(value a, value b)
{
  CAMLparam2(a, b);
  CAMLlocal1(pair);
  pair = caml_alloc(2, 0);
  Store_field(pair, 0, a);
  Store_field(pair, 1, b);
  CAMLreturn(pair);
}
\end{lstlisting}

The first macro \texttt{CAMLprim} makes the symbol visible from OCaml
code.

\paragraph{CAMLparam} \texttt{CAMLparam2(a,b)} expands to two other macros:

\begin{itemize}
\item \texttt{CAMLparam0()} saves the current local roots at the beginning
      of the function,
\item \texttt{CAMLxparam2(a,b)} setups a new block of roots with the
      addresses of \texttt{a} and \texttt{b}).
\end{itemize}

The local roots are in a linked list of pointer to OCaml values,
implemented by the \texttt{struct\ caml\_\_roots\_block} type, and
stored in the \texttt{caml\_local\_roots} global variable.

The job of the memory management macros is to make it as easy as
possible to register all local variables of type \texttt{value} in this
linked list and to remove them when returning from the function.

There should be only one \texttt{CAMLparam0()} in a function, but there
can be as many calls to \texttt{CAMLxparam} as needed. The variants from
\texttt{CAML(x)param1(x1)} to \texttt{CAML(x)param5(x1,x2,x3,x4,x5)} are
available, as well as \texttt{CAML(x)paramN(array,\ array\_size)} for
registering array of values.

\emph{CAMLlocal} The next macro call of interest is
\texttt{CAMLlocal1(pair)}. It expands to
\texttt{value\ pair\ =\ Val\_unit;\ CAMLxparam1(pair)}:

\begin{itemize}
\item it declares and initializes a local variable named \texttt{pair},
\item it adds its address to the set of local roots.
\end{itemize}

The next lines, the calls to \texttt{caml\_alloc} and
\texttt{Store\_field}, are not directly related to the management of
roots. They deal with the construction of OCaml values -- assuming that
all variables have been registered properly.

The last macro is \texttt{CAMLreturn}. It restores the
\texttt{caml\_local\_roots} variable to the state saved by
\texttt{CAMLparam0();}.

The code above can be desugared to the equivalent:

\begin{lstlisting}[]
CAMLprim
value mk_pair_c_impl(value a, value b)
{
  // CAMLparam2(a, b);
  CAMLparam0();      // 1) save the state of local roots
  CAMLxparam2(a, b); // 2a) add &a and &b to local roots

  // CAMLlocal1(pair);
  value pair = Val_unit;
  CAMLxparam1(pair); // 2b) add &pair to local roots

  ...
  
  // CAMLreturn(pair);
  CAMLdrop;         // 3) restore the state of local roots
                     //    (forgetting &a, &b and &pair)
  return(pair);
}
\end{lstlisting}

In plain words, the three fundamental operations are:

\begin{itemize}
\item saving local roots,
\item registering roots,
\item restoring local roots.
\end{itemize}

\subsubsection{Carefully dealing with intermediate results}

To illustrate how easy it is to misuse this API, let us recall an
example from Dolan{[}caml-oxide{]}.

Lets imagine one wants to make a triplet as two nested pairs:

\begin{lstlisting}[]
let triplet x y z = (x,(y,z))
\end{lstlisting}

Armed with \texttt{mk\_pair\_c\_impl} and following the rules above, one
might be tempted to write:

\begin{lstlisting}[]
CAMLprim
value c_triplet(value x, value y, value z)
{
  CAMLparam3(x,y,z);
  CAMLlocal1(triplet);

  triplet = mk_pair_c_impl(x, mk_pair_c_impl(y, z));

  CAMLreturn(triplet);
}
\end{lstlisting}

But a bug lies in this implementation: the C compiler might have already
loaded the value of \texttt{x} (for instance, by copying it on the
stack) before the nested call to \texttt{mk\_pair\_c\_impl(y,\ z)}.

If this call triggers a compaction and \texttt{x} gets moved, the old,
and wrong, value of \texttt{x} will be passed to the outer call.

The correct version uses an intermediate variable for the temporary
value:

\begin{lstlisting}[]
CAMLprim
value c_triplet(value x, value y, value z)
{
  CAMLparam3(x,y,z);
  CAMLlocal2(intermediate, triplet);

  intermediate = mk_pair_c_impl(y, z);
  triplet = mk_pair_c_impl(x, intermediate);

  CAMLreturn(triplet);
}
\end{lstlisting}

To avoid bugs, calls to functions manipulating the OCaml memory should
be linearized as often as possible and temporary results stored in local
roots.

\section{value*, one level of indirection}

The first change we propose is to replace return types of type
\texttt{value} in FFI functions that allocates by a new argument of type
\texttt{value*}. While it looks like a minor change, this brings a lot
of simplifications to the rest of the code and prevents some undefined
behaviors.

\subparagraph{No risk of unexpected copy}

A tricky source of bug that we highlighted in the previous section was
that OCaml values get unexpectedly copied while being manipulated. With
this new level of indirection only the pointer gets copied. If the GC
comes and rewrites the roots, the pointer will still point to the right
one.

Looking at the operations involved in terms of lifetime, reading a value
from a root makes it ephemeral: the value is valid only until the next
OCaml allocation -- or simply undefined if an allocation can happen in
the same sequence-point.

When directly working with \texttt{value}, this operation is implicit.
Working with \texttt{value*}, the operation becomes explicit and forces
the developer to think about its effects -- they choose when to
dereference the pointer. In practice, this should only happen in the
primitive functions of the Garbage Collector, relieving the user of the
API from this burden.

\paragraph{Figure: Indirect API}

\begin{lstlisting}[]
// Allocating functions with the new API
void icaml_alloc(value *root, size_t size, int tag);
void icaml_copy_string(value *root, const char *string);
void icaml_alloc_string(value *root, size_t size);
...
    
// Non-allocating functions still work direct values
long iLong_val(value *x);
value iVal_long(long x);
int iget_tag(value *x);
value iget_field(value *root, int n);
void iset_field(value *root, int n, value *v);
void iset_field_long(value *root, int n, long v);
...
\end{lstlisting}

\subparagraph{Return values become explicit}

Now that all functions that interact with the garbage collector take
pointers, most offending code patterns become impossible to write:

\begin{itemize}
\item functions returning void simply cannot be nested,
\item nesting a call to a function returning \texttt{value} where a
      \texttt{value*} is expected requires taking the address of a temporary
      (\texttt{\&get\_field(x,\ 0)}), which is rejected by the C compiler.
\end{itemize}

To illustrate the benefits of this approach, let's take a look at how
our \texttt{triplet} would look like:

\begin{lstlisting}[]
static void mk_pair(value *result, value *a, value *b)
{
  icaml_assert(result != a, result != b);
  icaml_alloc(result, 2, 0);
  iset_field(result, 0, a);
  iset_field(result, 1, b);
}

CAMLprim
value caml_triplet(value x, value y, value z)
{
  CAMLparam3(x, y, z);
  CAMLlocal2(pair, result);
  mk_pair(&pair, &y, &z);
  mk_pair(&result, &x, &pair);
  CAMLreturn(result);
}
\end{lstlisting}

\subparagraph{No need to repeat roots.}

Another benefit of this approach is that callees no longer have the
responsibility of registering roots for their arguments.

With the existing OCaml API, any function receiving an argument of type
\texttt{value} has to register a corresponding root. There are as many
roots for a value as sub-routines calls with the value as argument.

\begin{lstlisting}[]
TODO: some example?
\end{lstlisting}

With the indirect approach it is the duty of whoever produced a
\texttt{value*} to register the root. The final consumer also has to be
careful when dereferencing the value, but this will generally be done by
a primitive function of the GC interface. On the other hand, the rest of
the binding code no longer has to be littered with root management.

\subparagraph{Dealing with immediate values}

A reader familiar with OCaml binding code might be worried that working
with immediate values (an integer directly stored in a \texttt{value})
becomes more complicated with our approach than the normal API.

Immediate values enjoy a lot of nice properties in the OCaml FFI. Since
they do not play with the memory graph of OCaml -- they don't reference
blocks, they cannot be moved -- the rules for dealing with them are
relaxed: they don't have to be put in roots, they can be created without
triggering a garbage collection, etc.

However we argue these properties should not be exploited. That some
values can be represented without interacting with the GC is an
implementation detail. Being prepared for the ``worst case'' allow to
present an uniform interface without special-casing.

Still, the API provide the \texttt{iVal\_long} and
\texttt{iset\_field\_long} short-hands for cases that are known to be
safe.

\subsection{Safety of the indirect API}

Moving everything to pointers opened a new opportunity for incorrect
uses: aliasing. In the \texttt{triplet} case it would mean using the
\texttt{result} variable both as input and output in the same call:

\begin{lstlisting}[]
CAMLprim
value caml_triplet(value x, value y, value z)
{
  CAMLparam3(x, y, z);
  CAMLlocal1(result);
  mk_pair(&result, &y, &z);
  mk_pair(&result, &x, &result); // result is aliased!
  CAMLreturn(result);
}
\end{lstlisting}

While problematic indeed, this case is actually less worrying. The code
that dereferences roots can be instrumented to deal with that:

\begin{itemize}
\item by properly handling aliasing, for instance by ensuring that all
      arguments are read before any are written
\item by checking for this case and emitting a warning or failing, as
      illustrated by the assertions in the implementation of
      \texttt{mk\_pair} primitive. 
\end{itemize}

Actually thanks to the indirection we can go further than that. The
observation is that any well-formed argument of type \texttt{value*} is
actually a root. Native OCaml FFI is \texttt{value}-centric: functions
directly take and produce values. Our rewriting make it
\texttt{root}-centric: functions receive roots and manipulate values
though them.

The main API function that seems to break the \texttt{value*} rule is
\texttt{iget\_field}, which returns a value. But that is precisely
because this value is not a root and could be pointing to an arbitrary
place in the OCaml heap. To make further use of this value we would have
to store it in a root.

While the connection between a root and its value is lost with native
OCaml FFI, we can access it at any time with the indirect API. Plugging
into GC infrastructure we can introspect the roots and provide a debug
variant of the FFI where the validity of values is checked dynamically.
This prevents a large class of misuse of GC API and delayed heap
corruption bugs.

\subparagraph{Discussion: where to add the indirection?}

Having made explicit the distinction between values (of type
\texttt{value}) and roots (of type \texttt{value*}) in the API, one
could wonder why our API makes use of roots in places where values would
be fine: the arguments to \texttt{iget\_field}, \texttt{iLong\_val},
etc.

We are not totally decided on this issue and might revisit this decision
in the future. However the ability to dynamically check for correct use
and the more explicit, safer-looking nature of the resulting code makes
us prefer the approach favoring root arguments.

Beside the slight increase in verbosity, we did not find any drawback to
this approach.

\section{Regions: dynamic allocation of roots}

To further simplify the API described above we propose to make
allocation of roots simpler.

The \texttt{CAMLparam} and \texttt{CAMLlocal} macros declare OCaml roots
with a static lifetime, known at compile-time. This is nice for
performance but puts more burden on the developer.

The semantics of these macros is hard to understand and some use cases
are not easily covered. As we already saw, returning values is tricky,
but storing temporary values in code controlled by an external framework
is even more problematic.

\paragraph{The need for root side-channels}

For the sake of the example, let's imagine that we need to sort some C
structures containing OCaml values. To achieve this the
\texttt{qsort\_r} from the C standard library seems appropriate. It
takes an array of user-defined structures and a custom comparison
operator in the form of a function pointer.

\begin{lstlisting}[]
struct item {
  my_c_type x;
  value v;
\;}

static int
c_comparator(const void *item1, const void *item2, void *comparator)
{
    value v1 = ((const struct item *)item1)->v;
    value v2 = ((const struct item *)item2)->v;
    value ml_comparator = *(value*)comparator;
    return Val_int(caml_callback2(ml_comparator,v1,v2));
}

void sort_ocaml_items(struct item *items, size_t count, value *comparator)
{
    qsort_r(items, count, sizeof(struct item),
            c_comparator, comparator);
}
\end{lstlisting}

\texttt{caml\_callback2} is a primitive function of native OCaml FFI
that allows to invoke an OCaml closure from C code (note: for the sake
of simplicity we do not deal with the case where the callback raises an
exception).

Because some OCaml code is being called in the middle of the comparator,
the garbage collector can get called in the middle of the sorting. Even
if we registered roots for all the values in this array, the
implementation of \texttt{qsort\_r} might have made copies that will not
be updated by the GC. More generally, rewriting the array in the middle
of the sort can lead to unexpected behaviors.

Because we know all the values of interest prior to calling
\texttt{qsort\_r}, a solution is to work with pointer to values. One
first allocates an array of roots and pass pointers to this array.
However in other situations, the set of roots might be determined
dynamically.

\subsection{Region-based management}

To let the developer dynamically manage the set of roots, we propose a
simple API that over-approximates the lifetime of local roots:

\begin{lstlisting}[]
typedef struct ... region_t;
void rcaml_enter(region_t *region);
void rcaml_leave(region_t *region);
value *rcaml_new_root(void);
#define rCAMLregion(...) ...
#define rCAMLreturn(p) ...
\end{lstlisting}

In this approach, we distinguish between external and helper functions:

\begin{itemize}
\item external functions are the ones that can be directly called from OCaml,
\item helper functions implement useful routines for binding foreign code.
\end{itemize}

The external functions are responsible for setting up the region while
helper functions assume that a region has already been setup. Mimicking
\texttt{CAMLparam...} macros, we provide some sugar for registering
parameters while setting up the region:

\begin{lstlisting}[]
value *pair_helper(value *a, value *b)
{
    value *v = rcaml_new_root();
    icaml_alloc(v, 2, 0);
    icaml_set_field(v, 0, a);
    icaml_set_field(v, 0, b);
    return v;
}

CAMLprim
value mk_pair(value a, value b)
{
    rCAMLregion(&a, &b);
    rCAMLreturn(pair_helper(&a, &b));
}

CAMLprim
value mk_triplet(value x, value y, value z)
{
    rCAMLregion(&x, &y, &z);
    rCAMLreturn(pair_helper(&x, pair_helper(&y, &z)));
}
\end{lstlisting}

Setting up a region introduces a new set of local roots that can grow
dynamically as new roots are requested. Leaving a region releases all
the roots at once.

\subsection{Sub-regions}

Assuming that all roots have the same life-time as the external
entrypoint works well if a fixed amount of work has to be done. However,
for long-running function (for instance, an event loop driven by
C-code), the over-approximation of lifetimes can be problematic. For
these cases, we allow the introduction of sub-regions, valid in a local
scope.

These sub-regions follow a stack discipline: they can be nested and are
released in the reverse order of their allocation.

\begin{lstlisting}[]
void rcaml_subenter(region_t *region);
void rcaml_subleave(region_t *region);
\end{lstlisting}

For instance, the following code avoids leaking roots while transforming
all the elements of an array:

\begin{lstlisting}[]
void fold_array(value *acc, value *array)
{
    region_t region;
    size_t count = iget_size(array);
    for (size_t i = 0; i < count; ++i)
    {
        rcaml_subenter(&region);
        value *item = rcaml_new_root();
        *item = iget_field(array, i);
        process_item(*acc, *item);
        rcaml_subleave(&region);
    }
}
\end{lstlisting}

Like normal regions, macros can be used to automate some of the
boilerplate.

\subsection{Releasing the lock in a region}

So far we demonstrated the use of region to allocate and manage OCaml
memory. The concept can also be applied to the converse: preventing
allocation and manipulation of OCaml memory in a given scope.

Although a multi-core runtime is being developed (TODO: Reference Dolan
\& KC MCGC), the vanilla OCaml runtime can only executes on a single
thread of execution. When multiple threads are in use, a lock is used by
the OCaml runtime to ensure that only one of them executes OCaml code at
any given time.

The C FFI provides an API for releasing the OCaml runtime in a given
scope of code.

\begin{lstlisting}[]
// Existing API
void caml_release_runtime_system(void);
void caml_acquire_runtime_system(void);
\end{lstlisting}

These APIs can be wrapped in corresponding
\texttt{rcaml\_\{acquire,release\}\_runtime\_system} functions, that
does additional bookkeeping to ensure proper use of regions:

\begin{itemize}
\item new roots cannot be allocated,
\item setting up normal regions is forbidden, but a special kind of region
      allow reacquiring the runtime,
\item while the currnet global region is released, dereferencing a value is
      forbidden, most helper functions won't work.
\end{itemize}

\begin{lstlisting}[]
// Wrapper releasing the runtime
void rcaml_release_runtime_system(void);
void rcaml_acquire_runtime_system(void);

// Wrapper locally reacquiring the runtime
void rcaml_reacquire_runtime_system(void);
void rcaml_rerelease_runtime_system(void);

// Macros
#define rcaml_without_ocaml...
#define rcaml_with_ocaml...
// Macros that allow writing code such as:
// rcaml_without_ocaml
// {
//   long_running_c1();
//   rcaml_with_ocaml
//   {
//     commit_result();
//   }
//   long_running_c2();
// }
\end{lstlisting}

All these restrictions can be tested dynamically. While no checks can be
done at compile-time, errors can be detected deterministically during
execution.

\subsection{Calling OCaml from region-managed code}

The last feature that needs some special care from the region API is the
ability to call OCaml closures from C code. When switching back to OCaml
code, the runtime marks a region as disabled: the roots it contains are
still reachable, but no new roots can be added to the region.

This helps detecting and handling a few unfortunate cases:

\begin{itemize}
\item When re-entering C code from OCaml deeper in the call stack, an
      entrypoint that forgot to setup a region could allocate from the outer
      region by mistake.
\item If we are unlucky, the OCaml thread scheduler could preempt the
      current thread and the re-entry would happen from another
      thread\ldots{} messing with the internal datastructures of the regions
      library. By intercepting calls, we can rely on the OCaml runtime lock
      to also protect region sections.
\item The OCaml closure could raise an exception. The native FFI deals with
      this situation by simply dropping roots from the local roots linked
      list: since the nodes allocated by \texttt{CAMLparam/local} macros are
      stored on the stack, when an exception is raised the local root and
      stack pointers are simply reset to their value before entering the C
      code.
\end{itemize}

\paragraph{Handling exceptions}

The OCaml native FFI provides two means for calling OCaml closures:

\begin{itemize}
\item the \texttt{CAMLcallback()} variants, that do not intercept
      exceptions. The C code will be aborted by directly jumping to the
      OCaml code that called an external function.
\item the \texttt{CAMLcallback\_exn()} variants, that tag the return value
      to distinguish exceptional case.
\end{itemize}

The return value of \texttt{CAMLcallback\_exn()} should be tested for
the exceptional case with \texttt{Is\_exception\_result} before resuming
normal execution.

Because our region management code needs to execute cleanup code when
leaving a scope, we forbid the former case. The user-code has to handle
the exceptional case without resorting to non-local control flow.

While it would have been possible to provide support for non-local
jumps, it did not made much sense in the Qt case: the binding is
implemented in C++, which allow arbitrary code to be executed when
leaving a scope. C++ exceptions are expressive enough to compose all our
requirements (non-local control flow, proper interaction with the
regions and with OCaml GC), but the bindings themselves did not need
such feature.

\section{Conclusion}

materializes roots as a different type, here \texttt{value*}
%%%%%%%%%%%%%%%%%%%%%%%%%%%%%%%%%%%%%%%%%%%%%%%%%%%%%%%%%%%%%%%%%%%%%%%%%%%%%%

\label{sect:bib}
\bibliography{biblio}

\end{document}

